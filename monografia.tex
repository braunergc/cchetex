% Modelo de monografia para o curso de Matemática do CCHE-Centro de Ciências Humanas e Exatas da Universidade Estadual da Paraíba. 

%A estrutura de arquivos foi criada e as customizações necessárias implementadas pelo prof. Dr. Brauner Gonçalves Coutinho a partir da classe abntex2.

% Este documento só deverá ser alterado para incluir ou excluir
% elementos pré e pós textuais. Use o comentário do latex (%) caso
% deseje excluir algum elemento.

% PREÂMBULO  =======================================================
%Pacotes e configurações

%% Copyright 2012-2014 by abnTeX2 group at http://abntex2.googlecode.com/ 
%%
%% This work may be distributed and/or modified under the
%% conditions of the LaTeX Project Public License, either version 1.3
%% of this license or (at your option) any later version.
%% The latest version of this license is in
%%   http://www.latex-project.org/lppl.txt
%% and version 1.3 or later is part of all distributions of LaTeX
%% version 2005/12/01 or later.
%%
%% This work has the LPPL maintenance status `maintained'.
%% 
%% The Current Maintainer of this work is the abnTeX2 team, led
%% by Lauro César Araujo. Further information are available on 
%% http://abntex2.googlecode.com/
%%
%% This work consists of the files abntex2-modelo-trabalho-academico.tex,
%% abntex2-modelo-include-comandos and abntex2-modelo-references.bib
%%

% ------------------------------------------------------------------------
% ------------------------------------------------------------------------
% abnTeX2: Modelo de Trabalho Academico (tese de doutorado, dissertacao de
% mestrado e trabalhos monograficos em geral) em conformidade com 
% ABNT NBR 14724:2011: Informacao e documentacao - Trabalhos academicos -
% Apresentacao
% ------------------------------------------------------------------------
% ------------------------------------------------------------------------

\documentclass[
	% -- opções da classe memoir --
	12pt,				% tamanho da fonte
	openright,			% capítulos começam em pág ímpar (insere página vazia caso preciso)
	oneside,			% para impressão em verso e anverso. Oposto a oneside
	a4paper,			% tamanho do papel. 
	% -- opções da classe abntex2 --
	chapter=TITLE,		% títulos de capítulos convertidos em letras maiúsculas
	%section=TITLE,		% títulos de seções convertidos em letras maiúsculas
	%subsection=TITLE,	% títulos de subseções convertidos em letras maiúsculas
	%subsubsection=TITLE,% títulos de subsubseções convertidos em letras maiúsculas
	sumario=tradicional,
	% -- opções do pacote babel --
	english,			% idioma adicional para hifenização
	spanish,			% idioma adicional para hifenização
	brazil  			% o último idioma é o principal do documento
	]{abntex2}

%comentar abaixo caso comentada linha 57

%\setsecnumdepth{subsubsection}
%\settocdepth{subsubsection}

\usepackage{./nao_editar/config/cchetex}

% ---
% Pacotes básicos 
% ---
\usepackage{lmodern}			% Usa a fonte Latin Modern			
%\usepackage{times}

%\usepackage[brazil]{babel}
%\addto{\captionsbrazil}{%
%  \renewcommand{\bibname}{REFER\^{E}NCIAS}
%}

\usepackage[T1]{fontenc}		% Selecao de codigos de fonte.
\usepackage[utf8]{inputenc}		% Codificacao do documento (conversão automática dos acentos)
\usepackage{lastpage}			% Usado pela Ficha catalográfica
\usepackage{indentfirst}		% Indenta o primeiro parágrafo de cada seção.
\usepackage{color}				% Controle das cores
\usepackage{graphicx} \graphicspath{{./figs/}}			% Inclusão de gráficos
\usepackage{microtype} 			% para melhorias de justificação
% ---
\usepackage{pdfpages}
% ---
% Pacotes adicionais, usados apenas no âmbito do Modelo Canônico do abnteX2
% ---
\usepackage{lipsum}				% para geração de dummy text
% ---

% deixar simples biblio
% \let\OLDthebibliography\thebibliography
% \renewcommand\thebibliography[1]{
%   \OLDthebibliography{#1}
%   \setlength{\parskip}{0pt}
%   \setlength{\itemsep}{0pt plus 0.3ex}
% }



\usepackage{url}
\def\UrlBreaks{\do\/\do-}
% ---
% Pacotes de citações
% ---
\usepackage[brazilian,hyperpageref]{backref}	 % Paginas com as citações na bibliografia
\usepackage[alf, abnt-emphasize=bf]{abntex2cite}	% Citações padrão ABNT
\usepackage{wallpaper}
\usepackage{multirow}
\usepackage{todonotes}

%\usepackage[subrefformat=parens,labelformat=parens]{subfig}

\usepackage[small]{caption}
\usepackage{subcaption}


%\mt{} marca texto dentro das chaves.
\usepackage{ulem}
\newcommand\mt{\bgroup\markoverwith
  {\textcolor{yellow}{\rule[-.5ex]{2pt}{2.5ex}}}\ULon}



% --- 
% CONFIGURAÇÕES DE PACOTES
% --- 

% ---
% Configurações do pacote backref
% Usado sem a opção hyperpageref de backref
\renewcommand{\backrefpagesname}{Citado na(s) página(s):~}

% Texto padrão antes do número das páginas
\renewcommand{\backref}{}
% Define os textos da citação
\renewcommand*{\backrefalt}[4]{
	\ifcase #1 %
		%Nenhuma citação no texto.% comentado caso de pura
	\or
		Citado na página #2.%
	\else
		%Citado #1 vezes nas páginas #2.%
		Citado nas páginas #2.%
	\fi}%
% ---

% ---
% Informações de dados para CAPA e FOLHA DE ROSTO
% ---
\titulo{Coloque o título do seu trabalho aqui: com subtítulo se houver}
\autor{Coloque o seu nome aqui!}
\areaconcentracao{Matemática aplicada}
\local{Monteiro}
\data{2017} 
\orientador{Prof. Dr. Coloque o nome do orientador aqui}

% ATENÇÃO: Para as titulações utilize:
% Me. para Mestre
% Ma. para Mestra
% Dr. para Doutor
% Dra. para Doutora

% ---
% Informações de dados para FOLHA DE APROVAÇÂO
% ---

\primeiroexaminador{Prof. Dr. Fulano de Tal}
\dadosprimeiroexaminador{Examinador externo (CCT/UFCG)}
\segundoexaminador{Prof. Ma. Fulana de Tal}
\dadossegundoexaminador{Examinador interno (CCHE/UEPB)}
\datadefesa{12/11/2017}

%LEIA COM ATENÇÃO AS LINHAS ABAIXO

% Retire o % do início da linha 31 se o seu trabalho deve conter uma seção descrevendo siglas
\newif\ifsiglas
%\siglastrue

% Retire o % do início da linha 35 se o seu trabalho deve conter uma seção descrevendo símbolos
\newif\ifsimbolos
%\simbolostrue

% Retire o % do início da linha 39 se o seu trabalho deve conter uma seção de anexos
\newif\ifanexos
%\anexostrue

% Retire o % do início da linha 43 se o seu trabalho deve conter uma seção de apêndices
\newif\ifapendices
%\apendicestrue

% Coloque o % do início da linha 47 se o seu trabalho não deve conter uma página com lista de Figuras
\newif\iffiguras
\figurastrue

% Coloque o % do início da linha 51 se o seu trabalho não deve conter uma página com lista de Tabelas
\newif\iftabelas
\tabelastrue

% Coloque o % do início da linha 53 se o seu trabalho não deve conter uma página com lista de Quadros
\newif\ifquadros
\quadrostrue

% ---
% Informações de dados para CAPA e FOLHA DE ROSTO
% ---

\instituicao{
  UNIVERSIDADE ESTADUAL DA PARAÍBA
\par
  CAMPUS VI - POETA PINTO DO MONTEIRO
\par
  CENTRO DE CIÊNCIAS HUMANAS E EXATAS
\par
  CURSO DE LICENCIATURA PLENA EM MATEMÁTICA}

%\instituicao{%
%  Universidade Estadual da Paraíba
%\par
%  Campus VI - Poeta Pinto do Monteiro
%\par
%  Centro de Ciências Humanas e Exatas
%\par  
%  Curso de Licenciatura Plena em Matemática}

\tipotrabalho{Monografia}
% O preambulo deve conter o tipo do trabalho, o objetivo, 
% o nome da instituição e a área de concentração 
\preambulo{Trabalho de Conclusão do Curso apresentado à coordenação do curso de Licenciatura em Matemática do Centro de Ciências Humanas e Exatas da Universidade Estadual da Paraíba, em cumprimento às exigências legais para a obtenção do título de Graduado no Curso de Licenciatura Plena em Matemática.}
% ---

% ---
% Configurações de aparência do PDF final

% alterando o aspecto da cor azul
\definecolor{blue}{RGB}{41,5,195}

% informações do PDF
\makeatletter

% No indice subseções com letras maiúsculas
% \let\oldcontentsline\contentsline
% \def\contentsline#1#2{%
%   \expandafter\ifx\csname l@#1\endcsname\l@section
%     \expandafter\@firstoftwo
%   \else
%     \expandafter\@secondoftwo
%   \fi
%   {%
%     \oldcontentsline{#1}{\MakeTextUppercase{#2}}%
%   }{%
%     \oldcontentsline{#1}{#2}%
%   }%
% }


\hypersetup{
     	%pagebackref=true,
		pdftitle={\@title}, 
		pdfauthor={\@author},
    	pdfsubject={\imprimirpreambulo},
	    pdfcreator={LaTeX with abnTeX2},
		pdfkeywords={abnt}{latex}{abntex}{abntex2}{trabalho acadêmico}, 
		colorlinks=true,       		% false: boxed links; true: colored links
    	linkcolor=black,          	% color of internal links
    	citecolor=black,        		% color of links to bibliography
    	filecolor=magenta,      		% color of file links
		urlcolor=black,
		bookmarksdepth=4
}
\makeatother
% --- 

% --- 
% Espaçamentos entre linhas e parágrafos 
% --- 

% O tamanho do parágrafo é dado por:
\setlength{\parindent}{1.3cm}

% Controle do espaçamento entre um parágrafo e outro:
\setlength{\parskip}{0.2cm}  % tente também \onelineskip

% ---
% Possibilita criação de Quadros e Lista de quadros.
% Ver https://github.com/abntex/abntex2/issues/176
%
\ifquadros

\newcommand{\quadroname}{Quadro}
\newcommand{\listofquadrosname}{Lista de quadros}

\newfloat[chapter]{quadro}{loq}{\quadroname}
\newlistof{listofquadros}{loq}{\listofquadrosname}
\newlistentry{quadro}{loq}{0}

% configurações para atender às regras da ABNT
\setfloatadjustment{quadro}{\centering}
\counterwithout{quadro}{chapter}
\renewcommand{\cftquadroname}{\quadroname\space} 
\renewcommand*{\cftquadroaftersnum}{\hfill--\hfill}

\setfloatlocations{quadro}{hbtp} % Ver https://github.com/abntex/abntex2/issues/176
% ---
\fi


% ---
% compila o indice
% ---
%\makeindex
% ---
\cftinserthook{toc}{APP}

% Pacotes, configurações e Definições para matemática pura.
\usepackage{amssymb}     % qed
\usepackage{amsthm}      % Teoremas
\usepackage{thmtools}    % Front end para amsthm (\declaretheorem)

\usepackage{amsfonts}

\setlist[enumerate]{leftmargin=16pt}

\declaretheorem[style=definition,name=Definição,parent=chapter]{defi}
\declaretheorem[style=definition,name=Observação,parent=chapter]{obs}
\declaretheorem[style=definition,name=Corolário,parent=chapter]{coro}
\declaretheorem[style=plain,name=Teorema,parent=chapter]{teo}
\declaretheorem[style=plain,name=Proposição,parent=chapter]{prop}
\declaretheorem[style=remark,name=Demonstração,numbered=no,qed=$\blacksquare$]{prova}
\declaretheorem[style=plain,name=Lema,parent=teo]{lema}
\declaretheorem[style=definition,name=Exemplo,parent=chapter]{ex}
\declaretheorem[style=definition,name=Aplicação,parent=chapter]{apl}
\declaretheorem[style=plain,name=Axioma]{axioma}

\addto\captionsbrazil{
%% ajusta nomes padroes do babel
\renewcommand{\bibname}{REFER\^ENCIAS}
%\renewcommand{\indexname}{\'INDICE}
%\renewcommand{\contentsname}{SUM\'ARIO}
%\renewcommand{\listfigurename}{LISTA DE ILUSTRA\c C\~OES}
%\renewcommand{\listtablename}{LISTA DE TABELAS}
%\renewcommand{\agradecimentosname}{AGRADECIMENTOS}
%\renewcommand{\listadesiglasname}{LISTA DE ABREVIATURAS E SIGLAS}
%\renewcommand{\listadesimbolosname}{LISTA DE S\'IMBOLOS}
%% ajusta nomes usados com a macro \autoref
%\renewcommand{\pageautorefname}{p\'agina}
%\renewcommand{\sectionautorefname}{se{\c c}\~ao}
%\renewcommand{\subsectionautorefname}{subse{\c c}\~ao}
%\renewcommand{\paragraphautorefname}{par\'agrafo}
%\renewcommand{\subsubsectionautorefname}{subse{\c c}\~ao}
}

% Início do documento
\setlength {\marginparwidth }{2cm}

% \renewcommand{\thesubfigure}{(\alph{subfigure})}   % <---
% \newcommand\figref[1]{Fig.~\ref{#1}}

\begin{document} 

% Retira espaço extra obsoleto entre as frases.
\frenchspacing

% ELEMENTOS PRÉ-TEXTUAIS ==========================================
% \pretextual

% Capa
\imprimircapa

% Folha de rosto (* indica que há a ficha bibliográfica)
\imprimirfolhaderosto*

%\input{./nao_editar/ficha}

\includepdf[pages=-]{ficha.pdf}

%\input{pre/__errata} %se houver, descomente essa linha
%Instruções para adicionar a página com as assinaturas que você terá após a defesa:
% 1 - Escaneie a folha de aprovação com as assinaturas. Certifique-se que está com boa qualidade;
% 2 - O arquivo deve estar em formato pdf. Caso não esteja realize a conversão e renomeie o arquivo para aprovacao.pdf (sem cedilha e nem acento); 
% 3 - Coloque o arquivo aprovacao.pdf na pasta figs;
% 4 - Apague o % do início da linha 8;
% 5 - Digite um % no início da linha 10.

%\includepdf[pages=-]{aprovacao.pdf}

\input{nao_editar/aprovacao_em_branco}

%Dedicatória
\begin{dedicatoria} \vspace*{\fill} \centering \noindent \textit{
Este trabalho é dedicado às crianças adultas que,
quando pequenas, sonharam em se tornar cientistas.

} \vspace*{\fill} \end{dedicatoria}

%Agradecimentos
\begin{agradecimentos} 
\input{./editar/agradecimentos}
\end{agradecimentos}

%Epigrafe
\begin{epigrafe} \vspace*{\fill} \begin{flushright}	\textit{
\begin{minipage}{.8\textwidth}
\begin{flushright}
\input{./editar/epigrafe}
\end{flushright}
\end{minipage}
} \end{flushright} \end{epigrafe}

%                  _                               
%                 | |                              
% _ __   ___  _ __| |_ _   _  __ _ _   _  ___  ___ 
%| '_ \ / _ \| '__| __| | | |/ _` | | | |/ _ \/ __|
%| |_) | (_) | |  | |_| |_| | (_| | |_| |  __/\__ \
%| .__/ \___/|_|   \__|\__,_|\__, |\__,_|\___||___/
%| |                          __/ |                
%|_|                         |___/                 
\setlength{\absparsep}{18pt} % ajusta o espaçamento dos parágrafos do resumo
\begin{resumo}[RESUMO]

\input{editar/resumo_por}

\end{resumo}

% _             _           
%(_)           | |          
% _ _ __   __ _| | ___  ___ 
%| | '_ \ / _` | |/ _ \/ __|
%| | | | | (_| | |  __/\__ \
%|_|_| |_|\__, |_|\___||___/
%          __/ |            
%         |___/             

\begin{resumo}[ABSTRACT]
 \begin{otherlanguage*}{english}

% Substitua o texto abaixo pelo seu resumo em inglês.

This is the english abstract.

% Substitua as palavras-chave em inglês abaixo pelas suas.

\textbf{Key-words}: latex. abntex. text editoration.

 \end{otherlanguage*}
\end{resumo}

%                             _           _ 
%                            | |         | |
%  ___  ___ _ __   __ _ _ __ | |__   ___ | |
% / _ \/ __| '_ \ / _` | '_ \| '_ \ / _ \| |
%|  __/\__ \ |_) | (_| | | | | | | | (_) | |
% \___||___/ .__/ \__,_|_| |_|_| |_|\___/|_|
%          | |                              
%          |_|                              

%\begin{resumo}[Resumen]
% \begin{otherlanguage*}{spanish}
%   Este es el resumen en español.
  
%   \textbf{Palabras clave}: latex. abntex. publicación de textos.
% \end{otherlanguage*}
%\end{resumo}
% --- 
 
\iffiguras
\pdfbookmark[0]{\listfigurename}{lof} \listoffigures* \cleardoublepage
\fi
\iftabelas
\pdfbookmark[0]{\listtablename}{lot} \listoftables* \cleardoublepage
\fi

% ---
% inserir lista de quadros
% ---
\ifquadros
\pdfbookmark[0]{\listofquadrosname}{loq}
\listofquadros*
\cleardoublepage
\fi
% ---


%Siglas
\ifsiglas
\begin{siglas} 
\input{editar/siglas} 
\end{siglas}
\fi


%Símbolos
\ifsimbolos
\begin{simbolos} 
\input{editar/simbolos} 
\end{simbolos}
\fi

\pdfbookmark[0]{\contentsname}{toc} 
\tableofcontents*
\cleardoublepage
%\cftinserthook{toc}{POST}


% ELEMENTOS TEXTUAIS ==============================================
\textual
% ----------------------------------------------------------
% Introdução (exemplo de capítulo sem numeração, mas presente no Sumário)
% ----------------------------------------------------------
%\chapter*[Introdução]{Introdução}
\chapter[Título da seção primária 
]{Título da seção primária}
%\addcontentsline{toc}{chapter}{INTRODUÇÃO}
% ----------------------------------------------------------

\section{Título da seção secundária }
\subsection{Título da seção terciária}
\subsubsection{Título da seção quaternária}
\subsubsubsection{Título da seção quinária}


\chapter[Resultados dos comandos]{Resultados dos comandos} \label{cap_exemplos}

Este capítulo mostra como usar diversos recursos da classe abntex2. Está transcrito de acordo com o original.

% ---
\section{Citações diretas}
\label{sec-citacao}
% ---
	
Utilize o ambiente \texttt{citacao} para incluir
citações diretas com mais de três linhas sim senhor:

\begin{citacao}
As citações diretas, no texto, com mais de três linhas, devem ser
destacadas com recuo de 4 cm da margem esquerda, com letra menor que a do texto
utilizado e sem as aspas. No caso de documentos datilografados, deve-se
observar apenas o recuo \cite[5.3]{NBR10520:2002}.
\end{citacao}

Use o ambiente assim:

\begin{verbatim}
\begin{citacao}
As citações diretas, no texto, com mais de três linhas [...] deve-se observar
apenas o recuo \cite[5.3]{NBR10520:2002}.
\end{citacao}
\end{verbatim}

O ambiente \texttt{citacao} pode receber como parâmetro opcional um nome de
idioma previamente carregado nas opções da classe (\autoref{sec-hifenizacao}). Nesse
caso, o texto da citação é automaticamente escrito em itálico e a hifenização é
ajustada para o idioma selecionado na opção do ambiente. Por exemplo:

\begin{verbatim}
\begin{citacao}[english]
Text in English language in italic with correct hyphenation.
\end{citacao}
\end{verbatim}

Tem como resultado:

\begin{citacao}[english]
Text in English language in italic with correct hyphenation.
\end{citacao}

\index{citações!simples}Citações simples, com até três linhas, devem ser
incluídas com aspas. Observe que em \LaTeX as aspas iniciais são diferentes das
finais: ``Amor é fogo que arde sem se ver''.

% ---
\section{Notas de rodapé}
% ---

As notas de rodapé são detalhadas pela NBR 14724:2011 na seção 5.2.1\footnote{As
notas devem ser digitadas ou datilografadas dentro das margens, ficando
separadas do texto por um espaço simples de entre as linhas e por filete de 5
cm, a partir da margem esquerda. Devem ser alinhadas, a partir da segunda linha
da mesma nota, abaixo da primeira letra da primeira palavra, de forma a destacar
o expoente, sem espaço entre elas e com fonte menor
\citeonline[5.2.1]{NBR14724:2011}.}\footnote{Caso uma série de notas sejam
criadas sequencialmente, o \abnTeX\ instrui o \LaTeX\ para que uma vírgula seja
colocada após cada número do expoente que indica a nota de rodapé no corpo do
texto.}\footnote{Verifique se os números do expoente possuem uma vírgula para
dividi-los no corpo do texto.}. 


% ---
\section{Tabelas}
% ---

\index{tabelas}A \autoref{tab-nivinv} é um exemplo de tabela construída em
\LaTeX.

\begin{table}[htb]
\ABNTEXfontereduzida
\caption[Níveis de investigação]{Níveis de investigação.}
\label{tab-nivinv}
\begin{tabular}{p{2.6cm}|p{6.0cm}|p{2.25cm}|p{3.40cm}}
  %\hline
   \textbf{Nível de Investigação} & \textbf{Insumos}  & \textbf{Sistemas de Investigação}  & \textbf{Produtos}  \\
    \hline
    Meta-nível & Filosofia\index{filosofia} da Ciência  & Epistemologia &
    Paradigma  \\
    \hline
    Nível do objeto & Paradigmas do metanível e evidências do nível inferior &
    Ciência  & Teorias e modelos \\
    \hline
    Nível inferior & Modelos e métodos do nível do objeto e problemas do nível inferior & Prática & Solução de problemas  \\
   % \hline
\end{tabular}
\legend{Fonte: \citeonline{van86}}
\end{table}

Já a \autoref{tabela-ibge} apresenta umla criada conforme o padrão do
\citeonline{ibge1993} requerido pelas normas da ABNT para documentos técnicos e
acadêmicos.

\begin{table}[htb]
\IBGEtab{%
  \caption{Um Exemplo de tabela alinhada que pode ser longa
  ou curta, conforme padrão IBGE.}%
  \label{tabela-ibge}
}{%
  \begin{tabular}{ccc}
  \toprule
   Nome & Nascimento & Documento \\
  \midrule \midrule
   Maria da Silva & 11/11/1111 & 111.111.111-11 \\
  \midrule 
   João Souza & 11/11/2111 & 211.111.111-11 \\
  \midrule 
   Laura Vicuña & 05/04/1891 & 3111.111.111-11 \\
  \bottomrule
\end{tabular}%
}{%
  \fonte{Produzido pelos autores.}%
  \nota{Esta é uma nota, que diz que os dados são baseados na
  regressão linear.}%
  \nota[Anotações]{Uma anotação adicional, que pode ser seguida de várias
  outras.}%
  }
\end{table}


% ---
\section{Figuras}
% ---

Figuras podem ser incorporadas de arquivos externos, como é o caso da
\autoref{fig_grafico}. Se a figura que ser incluída se tratar de um diagrama, um
gráfico ou uma ilustração que você mesmo produza, priorize o uso de imagens
vetoriais no formato PDF. Com isso, o tamanho do arquivo final do trabalho será
menor, e as imagens terão uma apresentação melhor, principalmente quando
impressas, uma vez que imagens vetorias são perfeitamente escaláveis para
qualquer dimensão. Nesse caso, se for utilizar o Microsoft Excel para produzir
gráficos, ou o Microsoft Word para produzir ilustrações, exporte-os como PDF e
os incorpore ao documento conforme o exemplo abaixo. No entanto, para manter a
coerência no uso de software livre (já que você está usando \LaTeX e \abnTeX),
teste a ferramenta \textsf{InkScape}\index{InkScape}
(\url{http://inkscape.org/}). Ela é uma excelente opção de código-livre para
produzir ilustrações vetoriais, similar ao CorelDraw\index{CorelDraw} ou ao Adobe
Illustrator\index{Adobe Illustrator}. De todo modo, caso não seja possível
utilizar arquivos de imagens como PDF, utilize qualquer outro formato, como
JPEG, GIF, BMP, etc. Nesse caso, você pode tentar aprimorar as imagens
incorporadas com o software livre \textsf{Gimp}\index{Gimp}
(\url{http://www.gimp.org/}). Ele é uma alternativa livre ao Adobe
Photoshop\index{Adobe Photoshop}.

\begin{figure}[htb]
	\caption{\label{fig_grafico}Gráfico produzido em Excel e salvo como PDF}
	\begin{center}
	    \includegraphics[scale=0.5]{Marca-da-UEPB}
	\end{center}
	\legend{Fonte: \citeonline[p. 24]{araujo2012}}
\end{figure}

% ---
\subsection{Figuras em \emph{minipages}}
% ---

\emph{Minipages} são usadas para inserir textos ou outros elementos em quadros
com tamanhos e posições controladas. Veja o exemplo da
\autoref{fig_minipage_imagem1} e da \autoref{fig_minipage_grafico2}.

\begin{figure}[b]
 \label{teste}
 \centering
  \begin{minipage}{0.4\textwidth}
    \centering
    \caption{Imagem 1 da minipage.} \label{fig_minipage_imagem1}
    \includegraphics[scale=0.9]{Marca-da-UEPB}
    \legend{Fonte: Elaborada pelo autor, <ano>.}
  \end{minipage}
  \hfill
  \begin{minipage}{0.4\textwidth}
    \centering
    \caption{Grafico 2 da minipage} \label{fig_minipage_grafico2}
    \includegraphics[scale=0.2]{figs/Marca-da-UEPB.png}
    \legend{Elaborada pelo autor, <ano>.}
  \end{minipage}
\end{figure}

Observe que, segundo a \citeonline[seções 4.2.1.10 e 5.8]{NBR14724:2011}, as
ilustrações devem sempre ter numeração contínua e única em todo o documento:

\begin{citacao}
Qualquer que seja o tipo de ilustração, sua identificação aparece na parte
superior, precedida da palavra designativa (desenho, esquema, fluxograma,
fotografia, gráfico, mapa, organograma, planta, quadro, retrato, figura,
imagem, entre outros), seguida de seu número de ordem de ocorrência no texto,
em algarismos arábicos, travessão e do respectivo título. Após a ilustração, na
parte inferior, indicar a fonte consultada (elemento obrigatório, mesmo que
seja produção do próprio autor), legenda, notas e outras informações
necessárias à sua compreensão (se houver). A ilustração deve ser citada no
texto e inserida o mais próximo possível do trecho a que se
refere. \cite[seções 5.8]{NBR14724:2011}
\end{citacao}

\subsection{Múltiplas figuras}

Quando se deseja agrupar múltiplas figuras em uma só, de modo que apareça um único nome, pode se proceder como mostra a \autoref{fig:3cenas}. Em um caso assim, a referência no texto pode ser feita para de duas maneiras:

\begin{itemize}
    \item Para chamar a figura que contém todas as 3: \autoref{fig:3cenas};
    \item Para fazer referência individualmente, para a terceira figura, por exemplo: \autoref{fig:3cenas}(\subref{fig:cena3});
    \item Outra opção para a terceira figura: \autoref{fig:3cenas}(c).
\end{itemize}

Outro exemplo pode ser visto na \autoref{fig:3cenas2}, com um posicionamento lado a lado. Neste caso, o tamanho do texto pode alterar o alinhamento vertical das figuras. O ideal é deixar os textos das figuras laterais com o mesmo comprimento, aproximadamente. 

Em ambos os casos é preciso configurar o trecho do código que regula a largura das imagens: \verb|{0.7\textwidth}|.

\begin{figure}
     \centering
     \caption{Capturas de tela mostrando as três cenas simuladas.}
     \begin{subfigure}[b]{0.7\textwidth}
         \centering 
        \includegraphics[width=\textwidth]{figura-exemplo.png}
         \caption{Caso 1: Terra até a Lua.}
         \label{fig:cena1}
     \end{subfigure}
     \\
     \begin{subfigure}[b]{0.7\textwidth}
         \centering
         \includegraphics[width=\textwidth]{figura-exemplo.png}
         \caption{Caso 2: Terra até Marte.}
         \label{fig:cena2}
     \end{subfigure}
     \\
     \begin{subfigure}[b]{0.7\textwidth}
         \centering
         \includegraphics[width=\textwidth]{figura-exemplo.png}
         \caption{Caso 3: Sol até a Terra, passando por Mercúrio e Vênus.}
         \label{fig:cena3}
     \end{subfigure}
    \label{fig:3cenas}
    \legend {Fonte: Elaborada pelo autor, <ano>.}
\end{figure}

\begin{figure}[h!]
     \centering
     \caption{Explicação do experimento de Fizeau.}
     \begin{subfigure}[b]{0.44\textwidth}
         \centering
         \includegraphics[width=\textwidth]{figura-exemplo.png}
         \caption{A Luz emitida pela fonte atinge o espelho a cerca de 9km de distância.}
         \label{fig:im1}
     \end{subfigure}\\
     \begin{subfigure}[b]{0.44\textwidth}
         \centering
         \includegraphics[width=\textwidth]{figura-exemplo.png}
         \caption{A reflexão da luz percorre a mesma distância de volta e pode ser observada por meio de um espelho.}
         \label{fig:im2}
     \end{subfigure}\qquad
     \begin{subfigure}[b]{0.44\textwidth}
         \centering
         \includegraphics[width=\textwidth]{figura-exemplo.png}
         \caption{Dependendo da velocidade de rotação, o reflexo é interrompido ou pode passar pelo próximo entalhe da roda.}
         \label{fig:im3}
     \end{subfigure}
    \label{fig:3cenas2}
    \legend{Fonte: Elaborada pelo autor, <ano>.}
\end{figure}

% ---
\section{Expressões matemáticas}
% ---

\index{expressões matemáticas}Use o ambiente \texttt{equation} para escrever
expressões matemáticas numeradas:

\begin{equation}
  \forall x \in X, \quad \exists \: y \leq \epsilon
\end{equation}

Escreva expressões matemáticas entre \$ e \$, como em $ \lim_{x \to \infty}
\exp(-x) = 0 $, para que fiquem na mesma linha.

Também é possível usar colchetes para indicar o início de uma expressão
matemática que não é numerada.

\[
\left|\sum_{i=1}^n a_ib_i\right|
\le
\left(\sum_{i=1}^n a_i^2\right)^{1/2}
\left(\sum_{i=1}^n b_i^2\right)^{1/2}
\]

Consulte mais informações sobre expressões matemáticas em
\url{https://code.google.com/p/abntex2/wiki/Referencias}.

% ---
\section{Enumerações: alíneas e subalíneas}
% ---

\index{alíneas}\index{subalíneas}\index{incisos}Quando for necessário enumerar
os diversos assuntos de uma seção que não possua título, esta deve ser
subdividida em alíneas \cite[4.2]{NBR6024:2012}:

\begin{alineas}

  \item os diversos assuntos que não possuam título próprio, dentro de uma mesma
  seção, devem ser subdivididos em alíneas; 
  
  \item o texto que antecede as alíneas termina em dois pontos;
  \item as alíneas devem ser indicadas alfabeticamente, em letra minúscula,
  seguida de parêntese. Utilizam-se letras dobradas, quando esgotadas as
  letras do alfabeto;

  \item as letras indicativas das alíneas devem apresentar recuo em relação à
  margem esquerda;

  \item o texto da alínea deve começar por letra minúscula e terminar em
  ponto-e-vírgula, exceto a última alínea que termina em ponto final;

  \item o texto da alínea deve terminar em dois pontos, se houver subalínea;

  \item a segunda e as seguintes linhas do texto da alínea começa sob a
  primeira letra do texto da própria alínea;
  
  \item subalíneas \cite[4.3]{NBR6024:2012} devem ser conforme as alíneas a
  seguir:

  \begin{alineas}
     \item as subalíneas devem começar por travessão seguido de espaço;

     \item as subalíneas devem apresentar recuo em relação à alínea;

     \item o texto da subalínea deve começar por letra minúscula e terminar em
     ponto-e-vírgula. A última subalínea deve terminar em ponto final, se não
     houver alínea subsequente;

     \item a segunda e as seguintes linhas do texto da subalínea começam sob a
     primeira letra do texto da própria subalínea.
  \end{alineas}
  
  \item no \abnTeX\ estão disponíveis os ambientes \texttt{incisos} e
  \texttt{subalineas}, que em suma são o mesmo que se criar outro nível de
  \texttt{alineas}, como nos exemplos à seguir:
  
  \begin{incisos}
    \item \textit{Um novo inciso em itálico};
  \end{incisos}
  
  \item Alínea em \textbf{negrito}:
  
  \begin{subalineas}
    \item \textit{Uma subalínea em itálico};
    \item \underline{\textit{Uma subalínea em itálico e sublinhado}}; 
  \end{subalineas}
  
  \item Última alínea com \emph{ênfase}.
  
\end{alineas}

% ---
\section{Inclusão de outros arquivos}\label{sec-include}
% ---

É uma boa prática dividir o seu documento em diversos arquivos, e não
apenas escrever tudo em um único. Esse recurso foi utilizado neste
documento. Para incluir diferentes arquivos em um arquivo principal,
de modo que cada arquivo incluído fique em uma página diferente, utilize o
comando:

\begin{verbatim}
   \include{documento-a-ser-incluido}      % sem a extensão .tex
\end{verbatim}

Para incluir documentos sem quebra de páginas, utilize:

\begin{verbatim}
   \input{documento-a-ser-incluido}      % sem a extensão .tex
\end{verbatim}

% ---
\section{Remissões internas ou referências cruzadas}
% ---

Ao nomear a \autoref{tab-nivinv}, apresentamos um
exemplo de remissão interna (ou referência cruzada), que também pode ser feita quando indicamos o
\autoref{cap_exemplos}, que tem o nome \emph{\nameref{cap_exemplos}}. O número
do capítulo indicado é \ref{cap_exemplos}, que se inicia à
\autopageref{cap_exemplos}\footnote{O número da página de uma remissão pode ser
obtida também assim:
\pageref{cap_exemplos}.}.
Veja a \autoref{sec-divisoes} para outros exemplos de remissões internas entre
seções, subseções e subsubseções.

O código usado para produzir o texto desta seção é:

\begin{verbatim}
Ao nomear a \autoref{tab-nivinv}, apresentamos um
exemplo de remissão interna, que também pode ser feita quando indicamos o
\autoref{cap_exemplos}, que tem o nome \emph{\nameref{cap_exemplos}}. O número
do capítulo indicado é \ref{cap_exemplos}, que se inicia à
\autopageref{cap_exemplos}\footnote{O número da página de uma remissão pode ser
obtida também assim:
\pageref{cap_exemplos}.}.
Veja a \autoref{sec-divisoes} para outros exemplos de remissões internas entre
seções, subseções e subsubseções.
\end{verbatim}

% ---
\section{Divisões do documento: seção}\label{sec-divisoes}
% ---

Esta seção testa o uso de divisões de documentos. Esta é a
\autoref{sec-divisoes}. Veja a \autoref{sec-divisoes-subsection}.

\subsection{Divisões do documento: subseção}\label{sec-divisoes-subsection}

Isto é uma subseção. Veja a \autoref{sec-divisoes-subsubsection}, que é uma
\texttt{subsubsection} do \LaTeX, mas é impressa chamada de ``subseção'' porque
no Português não temos a palavra ``subsubseção''.

\subsubsection{Divisões do documento: subsubseção}
\label{sec-divisoes-subsubsection}

Isto é uma subsubseção.

\subsubsection{Divisões do documento: subsubseção}

Isto é outra subsubseção.

\subsection{Divisões do documento: subseção}\label{sec-exemplo-subsec}

Isto é uma subseção.

\subsubsection{Divisões do documento: subsubseção}

Isto é mais uma subsubseção da \autoref{sec-exemplo-subsec}.


\subsubsubsection{Esta é uma subseção de quinto
nível}\label{sec-exemplo-subsubsubsection}

Esta é uma seção de quinto nível. Ela é produzida com o seguinte comando:

\begin{verbatim}
\subsubsubsection{Esta é uma subseção de quinto
nível}\label{sec-exemplo-subsubsubsection}
\end{verbatim}

\subsubsubsection{Esta é outra subseção de quinto nível}\label{sec-exemplo-subsubsubsection-outro}

Esta é outra seção de quinto nível.

% ---
\section{Este é um exemplo de nome de seção longo. Ele deve estar
alinhado à esquerda e a segunda e demais linhas devem iniciar logo abaixo da
primeira palavra da primeira linha}
% ---

Isso atende à norma dede \citeonline[seções de 5.2.2 a 5.2.4]{NBR14724:2011} 
 e \citeonline[seções de 3.1 a 3.8]{NBR6024:2012}.

% ---
\section{Referências bibliográficas}
% ---

A formatação das referências bibliográficas conforme as regras da ABNT são um
dos principais objetivos do \abnTeX. Consulte os manuais
\citeonline{abntex2cite} e \citeonline{abntex2cite-alf} para obter informações
sobre como utilizar as referências bibliográficas.

%-
\subsection{Acentuação de referências bibliográficas}
%-

Normalmente não há problemas em usar caracteres acentuados em arquivos
bibliográficos (\texttt{*.bib}). Porém, como as regras da ABNT fazem uso quase
abusivo da conversão para letras maiúsculas, é preciso observar o modo como se
escreve os nomes dos autores. Na ~\autoref{tabela-acentos} você encontra alguns
exemplos das conversões mais importantes. Preste atenção especial para `ç' e `í'
que devem estar envoltos em chaves. A regra geral é sempre usar a acentuação
neste modo quando houver conversão para letras maiúsculas.

\begin{table}[htbp]
\caption{Tabela de conversão de acentuação.}
\label{tabela-acentos}

\begin{center}
\begin{tabular}{ll}\hline\hline
acento & \textsf{bibtex}\\
à á ã & \verb+\`a+ \verb+\'a+ \verb+\~a+\\
í & \verb+{\'\i}+\\
ç & \verb+{\c c}+\\
\hline\hline
\end{tabular}
\legend{Fonte: elaborada pelo autor, <ano>.}
\end{center}
\end{table}

Exemplo de um quadro pode ser visto no Quadro \ref{quadro_exemplo}.

\begin{quadro}[htb]
\caption{\label{quadro_exemplo}Exemplo de quadro}
\begin{tabular}{|c|c|c|c|}
	\hline
	\textbf{Pessoa} & \textbf{Idade} & \textbf{Peso} & \textbf{Altura} \\ \hline
	Marcos & 26    & 68   & 178    \\ \hline
	Ivone  & 22    & 57   & 162    \\ \hline
	...    & ...   & ...  & ...    \\ \hline
	Sueli  & 40    & 65   & 153    \\ \hline
\end{tabular}
\legend{Fonte: elaborada pelo autor, <ano>.}
\end{quadro}


\chapter[Exemplo de capítulo com notação matemática]{Exemplo de capítulo com notação matemática}

Este capítulo contém alguns exemplos de ambientes matemáticos.

\section{Como criar uma definição}

\begin{defi}
Uma Topologia sobre o conjunto $X$ é uma família $\tau
\subset \beta(X)$ que satisfaz $ \mathbb{S} $:
\begin{enumerate}
 \item$\emptyset,X$ pertencem a $\tau;$
 \item Seja  $\{A_{\alpha} \in \tau /\alpha \in \Gamma\}$
 uma família arbitrária de subconjuntos de  $\tau$ então, a união dos elementos dessa família ainda pertence a  $\tau$, ou seja, $\bigcup \limits_{\alpha \in \Gamma} A_{\alpha}\in\tau;$
 \item Sejam $B_{1},B_{2},B_{3},..., B_{n} \in \tau$ uma família finita de subconjuntos de $\tau$, com $n \geq 1$, então
 interseção dessa família ainda pertence a $\tau$, ou seja, $\bigcap\limits_{i=1}^{n} B_{i}\in \tau.$
\end{enumerate}
\end{defi}

\section{Como criar uma observação}

\begin{obs}

\item Os elementos da Topologia  $\tau$ caracterizam os Conjuntos Abertos de $X.$

\item O par ordenado $(X,\tau)$ é chamado de Espaço Topológico.

\item Sim, de fato:

\item i) $\emptyset, X \in \tau$, pois são subconjuntos de $X;$

\item ii) Seja \{$J_{\lambda}\}_{\lambda\in L}$ temos que$J_{\lambda}\in \tau$ então $\bigcup\limits_{\lambda\in L} J_{\lambda}$ é
    um subconjunto de $X$ e portanto, pertence a $\tau;$

\item iii) Sejam $J_{1},J_{2},...J_{n} \in \tau$ temos que $J_{1}\bigcap J_{2}\bigcap ...J_{n}$ é um suconjunto de $X$ e portanto pertence a $\tau.$

\end{obs}

\section{Como criar um exemplo}

\begin{ex}{Topologia Euclidiana ou Usual} - $\tau_{us}$

Essa topologia é formada por intervalo do $\mathbb{R}^{n}$
Os dois tipos mais estudados são a Topologia Euclidiana Usual na Reta (ou  de $\mathbb{R}$) e a Topologia Euclidiana Usual (ou do  no Plano Complexo (ou do $\mathbb{R}^{2}$).

Seja a topologia $\tau = \{A \subset \mathbb{R}\}, \emptyset\}$ onde $X = \mathbb{R}$, dizemos que $A \in \tau$, se e somente se, para todo $y \in A$, existe um intervalo aberto $(c,d)$ tal que
$y \in (c,d) \subset A.$

\item i) É fácil notar que $\mathbb{R}, \emptyset \in \tau;$

\item ii) Consideremos a família $\{A_{\alpha} \in \tau / \alpha \in\tau\},$
queremos mostrar que, $\bigcup\limits_{\alpha\in \Gamma} A_{\alpha}\in \tau.$

De fato, considerando $y \in \bigcup\limits_{\alpha\in \Gamma} A_{\alpha}$ ,
desse modo, existe $\alpha_{0} \in \Gamma$ tal que $y \in A_{\alpha_{0}} \in \tau$, assim, existe $(c,d)$ e
temos que $y\in (c,d) \subset A_{\alpha_{0}}\subset \bigcup\limits_{\alpha\in \Gamma}A_{\alpha}.$

\item iii) Consideremos $B_{1},B_{2}$ subconjuntos finitos de $\tau$, desse modo
tomemos $y \in B_{1}\bigcap B_{2}$, assim $y\in B_{1}$ e $y\in B_{2}$,
logo existem $(c_{1}, d_{1})$ e $(c_{2}, d_{2})$ tais que $y$
$\in0 (c_{1}, d_{1}) \subset B_{1}$ e $y\in (c_{2}, d_{2}) \subset B_{2}.$

Vamos denotar por $c = max(c_{1},c_{2})$ e $d = min(d_{1},d_{2})$, assim temos:

\begin{center}
 $y \in (c,d) \subset B_{1} \bigcap B_{2}$
\end{center}
 
Por indução temos  se $B_{1},B_{2},..., B_{n}$ são conjuntos finitos de $\tau$, então, $\bigcap\limits_{i=1}^{n}B_{i}\in\tau.$
\end{ex}

\section{Como criar um teorema}

\begin{teo}: Seja $(X,\tau)$ um espaço topológico e $ \mathfrak{F}$ a
família de conjuntos fechados, então:
\begin{description}
  \item[i)]$\emptyset, X$ pertencem a $ \mathfrak{F}$;
  \item[ii)]Considere $F_{1}, F_{2},..., F_{n}$ conjuntos fechados
  em $X$; então a união finita $\bigcup\limits_{i=1}^{n} F_{i}$ é fechada
  em $X$;
  \item[iii)]Considere $\{ F_{\alpha} \in \mathfrak{F} / \alpha \in
  \Gamma\}$ uma família de elementos arbitrários de $\mathfrak{F}$,
  então, a interseção arbitrária $\bigcap\limits_{\alpha \in \Gamma}F_{\alpha}$ pertence a $\mathfrak{F}.$
\end{description}
\end{teo}

\section{Como criar uma Prova}

 \begin{prova}~

 \item i) $\emptyset$ e $X$ pertencem a $\mathfrak{F}$, pois seus
   complementares pertencem a $\tau$.
 \item ii) Se $F_{i}$ é fechado para $i = 1,...,n$. Utilizando a Lei de Morgan, temos:
   \begin{center}
   $(\bigcup\limits_{i=1}^{n} F_{i})^{c} = \bigcap\limits_{i=1}^{n} (F_{i})^{c}$
   \end{center}
   
   Como a interseção $(F_{i})^{c}$ é  um aberto, pois, a interseção
   de conjuntos abertos é aberta, portanto, $\bigcup\limits_{i=1}^{n} F_{i}$ é fechada
   em $X.$
  \item iii) Seja $\{F_{\alpha \in \mathfrak{F} / \alpha \in
   \Gamma}\}$ uma família arbitrária de conjuntos fechados.

  Utilizando a Lei de Morgan, temos:
   \begin{center}
   $(\bigcap_{\alpha \in \Gamma} F_{\alpha})_{c} = \bigcup\limits_{\alpha \in
   \Gamma} F_{\alpha}^{c}$
   \end{center}
   
   Sabemos que $(F_{\alpha})^{c}$ é aberto, assim, a união de
   conjuntos abertos é aberta, então, podemos concluir que $\bigcap\limits_{\alpha \in
   \Gamma}F_{\alpha}$ é fechado em $X$.
 \end{prova}

% ---
% Conclusão (outro exemplo de capítulo sem numeração e presente no sumário)
% ---
\chapter[Conclusão]{Conclusão}
%\chapter*[Conclusão]{Conclusão}
%\addcontentsline{toc}{chapter}{CONCLUSÃO}
% ---

\lipsum[31-35]

% ELEMENTOS PÓS-TEXTUAIS ==========================================
\cftinserthook{toc}{POS}
\phantompart
\postextual
\addtocontents{toc}{\protect\vspace{-38pt}}
\bibliography{referencias}%\glossary
\ifapendices
% ----------------------------------------------------------
% Apêndices
% ----------------------------------------------------------

% ---
% Inicia os apêndices
% ---
%\begin{apendicesenv}
\apendices
%\appendix
%\appendixpage

% Imprime uma página indicando o início dos apêndices
% \partapendices

\chapter{Exemplo}

Este é um exemplo de apêndice.

%\end{apendicesenv}

\fi

\ifanexos
% ----------------------------------------------------------
% Anexos
% ----------------------------------------------------------

% ---
% Inicia os anexos
% ---
\anexos
%\begin{anexosenv}

% Imprime uma página indicando o início dos anexos
%\partanexos

\chapter{Exemplo}

Este é um exemplo de anexo.

%\end{anexosenv}
\fi

\phantompart
%\printindex

%---------------------------------------------------------------------
\end{document}
