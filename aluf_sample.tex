\documentclass[aspectratio=169]{beamer}
\usetheme{ALUF}

\usepackage[brazil]{babel}
\usepackage[utf8]{inputenc}
% \usepackage{palatino}
% \usepackage[T1]{fontenc}
\usepackage{lmodern}
\usepackage[expert]{mathdesign}
\usepackage[protrusion=true,expansion=true,tracking=true,kerning=true]{microtype}
\usepackage[alf]{abntex2cite}
\usepackage[document]{ragged2e}

\usepackage{enumerate}

% Pacotes, configurações e Definições para matemática pura.
\usepackage{amssymb}     % qed
\usepackage{amsthm}      % Teoremas
\usepackage{thmtools}    % Front end para amsthm (\declaretheorem)

\usepackage{amsfonts}

\setbeamertemplate{theorems}[numbered] % to number

\declaretheorem[style=definition,name=Definição,parent=section]{defi}
\declaretheorem[style=definition,name=Observação,parent=section]{obs}
\declaretheorem[style=definition,name=Corolário,parent=section]{coro}
\declaretheorem[style=plain,name=Teorema,parent=section]{teo}
\declaretheorem[style=plain,name=Proposição,parent=section]{prop}
\declaretheorem[style=remark,name=Demonstração,numbered=no,qed=$\blacksquare$]{prova}
\declaretheorem[style=plain,name=Lema,parent=teo]{lema}
\declaretheorem[style=definition,name=Exemplo,parent=section]{ex}
\declaretheorem[style=definition,name=Aplicação,parent=section]{apl}
\declaretheorem[style=plain,name=Axioma]{axioma}


\newcommand{\forceindent}{\leavevmode{\parindent=2em\indent}}

\newenvironment{citacao}%
  {\endgraf\justify
\quote}%
  {\newline\endquote}



\title{Escreva o título do trabalho aqui}
\subtitle{}
\author{Escreva o seu nome aqui}
\institute{\url{e-mail@dominio.edu}}
\date{Novembro de 2017}

\begin{document}

\begin{frame}[plain,t]
\titlepage
\end{frame}

\begin{frame}{Sumário}
\tableofcontents
\end{frame}

%=============================================================================================

\section{Introdução}
\begin{frame}{Aqui um exemplo de título de slide mais longo}
\framesubtitle{Pode-se criar um subtítulo com este comando}
\begin{itemize}
\item Primeiro item da lista:
	\begin{itemize}
	\item sub item 1
	\item sub item 2
	\end{itemize}
\item Segundo item da lista:
	\begin{enumerate}
	\item aqui um item de uma lista numerada
	\item aqui outro item de uma lista numerada	\end{enumerate}
\item Terceiro item da lista:
\end{itemize}
Abaixo uma lista do tipo descrição:
\begin{description}
	\item[Pesquisa] Busca científica pelo conhecimento
\end{description}
\end{frame}

\subsection{Awesome subsection que pode ter o nome longo}
\begin{frame}{Um título de slide aqui}

	Um exemplo de uma expressão matemática!

	\begin{equation}
		\begin{bmatrix}
	        \Phi_t \\
	        \Phi_{t+1} \\
	        \vdots \\
	        \Phi_{t+H}
	    \end{bmatrix}
	    ~=~
	    \begin{bmatrix}
	        \phi_t^1, \ldots, \phi_t^d \\
	        \phi_{t+1}^1, \ldots, \phi_{t+1}^d \\
	        \vdots \\
	        \phi_{t+H}^1, \ldots, \phi_{t+H}^d
	    \end{bmatrix}
		\label{eq:random}
	\end{equation}

\end{frame}





\subsection{Assim se define uma subseção}

\begin{frame}{Blocos}

\begin{defi}[Exemplo de definição com nome]
Hello World
\end{defi}

\begin{teo}[Teorema de Fermat]
$a^n + b^n = c^n, n \leq 2$
\end{teo}

%Exemplo de teorema sem título
%\begin{teo}[]
%$abc$
%\end{teo}

\end{frame}



\begin{frame}{Exemplo mais complexo}
\begin{defi}
Uma Topologia sobre o conjunto $X$ é uma família $\tau
\subset \beta(X)$ que satisfaz $ \mathbb{S} $:
%\begin{enumerate}[label=\roman*)]
\begin{enumerate}[i]
 \item $\emptyset,X$ pertencem a $\tau;$
 \item Seja  $\{A_{\alpha} \in \tau /\alpha \in \Gamma\}$
 uma família arbitrária de subconjuntos de  $\tau$ então, a união dos elementos dessa família ainda pertence a  $\tau$, ou seja, $\bigcup \limits_{\alpha \in \Gamma} A_{\alpha}\in\tau;$
 \item Sejam $B_{1},B_{2},B_{3},..., B_{n} \in \tau$ uma família finita de subconjuntos de $\tau$, com $n \geq 1$, então
 interseção dessa família ainda pertence a $\tau$, ou seja, $\bigcap\limits_{i=1}^{n} B_{i}\in \tau.$
\end{enumerate}
\end{defi}
   
\end{frame}




\begin{frame}{Sim}
\begin{ex}
Considere $(X, \tau_{sier})$. Determine os conjuntos
fechados de $X$.

A $\tau_{sier}$ é chamada Topologia Sierpinsk e é caracterizada da
seguinte forma: $\tau_{sier} = \{\emptyset, X ,\{a\}\}$ com o conjunto $X = \{a,b\}$.

Logo, $X, \emptyset$ são Conjuntos Fechados de $X$, pois pertencem a
$\tau$. O conjunto $\{b\}$ é fechado em $X$, pois $\{b\}^{c} = \{a\}
\in \tau_{sier}$
\end{ex}
\end{frame}








\section{Another Section esta daqui tambem} % (fold)
\label{sec:another_section}

\begin{frame}
	\frametitle{Notação}
	\begin{defi}[Variável aleatória]
		Consider $\Omega, F, \mu$, with $\Omega$ being the set of events, $F$ the $\sigma$-algebra on $\Omega$ and some arbitrary measure $\mu$. Further consider an observation space $\Omega', F', \mu'$... A random variable is a deterministic function that 'transports/maps' events from $\Omega$ to $\Omega'$ and effectively induces a new measure $\mu'$. When $\mu'(\Omega') = 1$, it is a probability measure \cite{nakayama:98}.

	\end{defi}
	
\end{frame}




\begin{frame}{Demonstração}
 \begin{prova}
   \begin{enumerate}[i]
   %\begin{enumerate}[label=\roman*)]
\item $\emptyset$ e $X$ pertencem a $\mathfrak{F}$, pois seus
   complementares pertencem a $\tau$.
   \item Se $F_{i}$ é fechado para $i = 1,...,n$. Utilizando a Lei de Morgan, temos:
   \begin{center}
   $(\bigcup\limits_{i=1}^{n} F_{i})^{c} = \bigcap\limits_{i=1}^{n} (F_{i})^{c}$
   \end{center}
   Como a interseção $(F_{i})^{c}$ é  um aberto, pois, a interseção
   de conjuntos abertos é aberta, portanto, $\bigcup\limits_{i=1}^{n} F_{i}$ é fechada
   em $X.$
   \item Seja $\{F_{\alpha \in \mathfrak{F} / \alpha \in
   \Gamma}\}$ uma família arbitrária de conjuntos fechados.
   \end{enumerate}
 \end{prova}    
\end{frame}





\begin{frame}{Citação}
Exemplo de citação:
\begin{citacao}
Um programa é uma sequência de códigos (comandos) escritos de forma conveniente para solucionar um problema ou visualizar um resultado previsto. A essa sequência estruturada e lógica de códigos, damos o nome de algorítmo e para que o mesmo seja compilado é preciso que ele obedeça à estrutura da linguagem trabalhada.
\end{citacao}

Acima foi um exemplo de citação.
\end{frame}


\begin{frame}{Referências}
\bibliography{referencias}
\end{frame}

% section another_section (end)


%\ThankYouFrame

\end{document}
