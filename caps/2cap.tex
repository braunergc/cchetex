\chapter[Exemplo de capítulo com notação matemática]{Exemplo de capítulo com notação matemática}

Este capítulo contém alguns exemplos de ambientes matemáticos.

\section{Como criar uma definição}

\begin{defi}
Uma Topologia sobre o conjunto $X$ é uma família $\tau
\subset \beta(X)$ que satisfaz $ \mathbb{S} $:
\begin{enumerate}
 \item$\emptyset,X$ pertencem a $\tau;$
 \item Seja  $\{A_{\alpha} \in \tau /\alpha \in \Gamma\}$
 uma família arbitrária de subconjuntos de  $\tau$ então, a união dos elementos dessa família ainda pertence a  $\tau$, ou seja, $\bigcup \limits_{\alpha \in \Gamma} A_{\alpha}\in\tau;$
 \item Sejam $B_{1},B_{2},B_{3},..., B_{n} \in \tau$ uma família finita de subconjuntos de $\tau$, com $n \geq 1$, então
 interseção dessa família ainda pertence a $\tau$, ou seja, $\bigcap\limits_{i=1}^{n} B_{i}\in \tau.$
\end{enumerate}
\end{defi}

\section{Como criar uma observação}

\begin{obs}

\item Os elementos da Topologia  $\tau$ caracterizam os Conjuntos Abertos de $X.$

\item O par ordenado $(X,\tau)$ é chamado de Espaço Topológico.

Sim, de fato:
\item[i)]$\emptyset, X \in \tau$, pois são subconjuntos de $X;$
\item[ii)]Seja \{$J_{\lambda}\}_{\lambda\in L}$ temos que$J_{\lambda}\in \tau$ então $\bigcup\limits_{\lambda\in L} J_{\lambda}$ é
    um subconjunto de $X$ e portanto, pertence a $\tau;$
\item[iii)] Sejam $J_{1},J_{2},...J_{n} \in \tau$ temos que $J_{1}\bigcap J_{2}\bigcap ...J_{n}$ é um suconjunto de $X$ e portanto pertence a $\tau.$

\end{obs}

\section{Como criar um exemplo}

\begin{ex}{Topologia Euclidiana ou Usual} - $\tau_{us}$

Essa topologia é formada por intervalo do $\mathbb{R}^{n}$
Os dois tipos mais estudados são a Topologia Euclidiana Usual na Reta (ou  de $\mathbb{R}$) e a Topologia Euclidiana Usual (ou do  no Plano Complexo (ou do $\mathbb{R}^{2}$).

Seja a topologia $\tau = \{A \subset \mathbb{R}\}, \emptyset\}$ onde $X = \mathbb{R}$, dizemos que $A \in \tau$, se e somente se, para todo $y \in A$, existe um intervalo aberto $(c,d)$ tal que
$y \in (c,d) \subset A.$

\item[i)] É fácil notar que $\mathbb{R}, \emptyset \in \tau;$

\item[ii)] Consideremos a família $\{A_{\alpha} \in \tau / \alpha \in\tau\},$
queremos mostrar que, $\bigcup\limits_{\alpha\in \Gamma} A_{\alpha}\in \tau.$

De fato, considerando $y \in \bigcup\limits_{\alpha\in \Gamma} A_{\alpha}$ ,
desse modo, existe $\alpha_{0} \in \Gamma$ tal que $y \in A_{\alpha_{0}} \in \tau$, assim, existe $(c,d)$ e
temos que $y\in (c,d) \subset A_{\alpha_{0}}\subset \bigcup\limits_{\alpha\in \Gamma}A_{\alpha}.$

\item[iii)] Consideremos $B_{1},B_{2}$ subconjuntos finitos de $\tau$, desse modo
tomemos $y \in B_{1}\bigcap B_{2}$, assim $y\in B_{1}$ e $y\in B_{2}$,
logo existem $(c_{1}, d_{1})$ e $(c_{2}, d_{2})$ tais que $y$
$\in0 (c_{1}, d_{1}) \subset B_{1}$ e $y\in (c_{2}, d_{2}) \subset B_{2}.$

Vamos denotar por $c = max(c_{1},c_{2})$ e $d = min(d_{1},d_{2})$, assim temos:

\begin{center}
 $y \in (c,d) \subset B_{1} \bigcap B_{2}$
\end{center}
 
Por indução temos  se $B_{1},B_{2},..., B_{n}$ são conjuntos finitos de $\tau$, então, $\bigcap\limits_{i=1}^{n}B_{i}\in\tau.$
\end{ex}

\section{Como criar um teorema}

\begin{teo}: Seja $(X,\tau)$ um espaço topológico e $ \mathfrak{F}$ a
família de conjuntos fechados, então:
\begin{description}
  \item[i)]$\emptyset, X$ pertencem a $ \mathfrak{F}$;
  \item[ii)]Considere $F_{1}, F_{2},..., F_{n}$ conjuntos fechados
  em $X$; então a união finita $\bigcup\limits_{i=1}^{n} F_{i}$ é fechada
  em $X$;
  \item[iii)]Considere $\{ F_{\alpha} \in \mathfrak{F} / \alpha \in
  \Gamma\}$ uma família de elementos arbitrários de $\mathfrak{F}$,
  então, a interseção arbitrária $\bigcap\limits_{\alpha \in \Gamma}F_{\alpha}$ pertence a $\mathfrak{F}.$
\end{description}
\end{teo}

\section{Como criar uma Prova}

 \begin{prova}
 \begin{enumerate}[label=\roman*)]
   \item $\emptyset$ e $X$ pertencem a $\mathfrak{F}$, pois seus
   complementares pertencem a $\tau$.
   \item [ii)] Se $F_{i}$ é fechado para $i = 1,...,n$. Utilizando a Lei de Morgan, temos:
   \begin{center}
   $(\bigcup\limits_{i=1}^{n} F_{i})^{c} = \bigcap\limits_{i=1}^{n} (F_{i})^{c}$
   \end{center}
   Como a interseção $(F_{i})^{c}$ é  um aberto, pois, a interseção
   de conjuntos abertos é aberta, portanto, $\bigcup\limits_{i=1}^{n} F_{i}$ é fechada
   em $X.$
   \item [iii)] Seja $\{F_{\alpha \in \mathfrak{F} / \alpha \in
   \Gamma}\}$ uma família arbitrária de conjuntos fechados.
   \end{enumerate}
   Utilizando a Lei de Morgan, temos:
   \begin{center}
   $(\bigcap_{\alpha \in \Gamma} F_{\alpha})_{c} = \bigcup\limits_{\alpha \in
   \Gamma} F_{\alpha}^{c}$
   \end{center}
   Sabemos que $(F_{\alpha})^{c}$ é aberto, assim, a união de
   conjuntos abertos é aberta, então, podemos concluir que $\bigcap\limits_{\alpha \in
   \Gamma}F_{\alpha}$ é fechado em $X$.
 \end{prova}
